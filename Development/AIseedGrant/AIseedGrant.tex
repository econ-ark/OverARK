
\documentclass[11pt,pdftex,letterpaper]{article}
\usepackage{hyperref}
            \usepackage{setspace}
            \usepackage[dvips,]{graphicx} %draft option suppresses graphics dvi display
%            \usepackage{lscape}
%            \usepackage{latexsym}
%            \usepackage{endnotes}
%            \usepackage{epsfig}
            \usepackage{amsmath}
%           \singlespace
            \setlength{\textwidth}{6.5in}
            \setlength{\textheight}{9in}
            \addtolength{\topmargin}{-\topmargin} 
            \setlength{\oddsidemargin}{0in}
            \setlength{\evensidemargin}{0in}
            \addtolength{\headsep}{-\headsep}
            \addtolength{\topskip}{-\topskip}
            \addtolength{\headheight}{-\headheight}
            \setcounter{secnumdepth}{2}
%            \renewcommand{\thesection}{\arabic{section}}
            % \renewcommand{\footnote}{\endnote}
            \newtheorem{proposition}{Proposition}
            \newtheorem{definition}{Definition}
            \newtheorem{lemma}{lemma}
            \newtheorem{corollary}{Corollary}
            \newtheorem{assumption}{Assumption}
            \newcommand{\Prob}{\operatorname{Prob}}
            \clubpenalty 5000
            \widowpenalty 5000
            \renewcommand{\baselinestretch}{1.35}
            \usepackage{amsmath}
            \usepackage{amsthm}
            \usepackage{amsfonts}
            \usepackage{amssymb}
            \usepackage{bbm}
            \usepackage{natbib}
            \newcommand{\N}{\mathbb{N}}
			\newcommand{\R}{\mathbb{R}}
			\newcommand{\E}{\mathbb{E}}
			\newcommand{\der}[2]{\frac{\text{d}#1}{\text{d}#2}}
			\newcommand{\pd}[2]{\frac{\partial#1}{\partial#2}}
						
\begin{document}
\thispagestyle{empty}

\begin{center}
	\textbf{Proposal for AI-Informed Discovery and Inquiry Seed Grant 2025 \\ September 30, 2024}
\end{center}

\vspace{1.5cm}

\noindent \textbf{Project Title:} A Deeper Dive on Deep-Learning Solution Methods for Heterogeneous Agents Macroeconomic Models

\vspace{1.5cm}

\noindent \textbf{Principal Investigator:} Professor Christopher D. Carroll (\texttt{ccarroll@jhu.edu}), Department of Economics, Krieger School of Arts and Sciences

\newpage
\setcounter{page}{1}

\begin{center}
	\textbf{Proposal Narrative}
      \end{center}

      \textbf{Statement of the Problem or Area of Investigation}. Economists have almost universally adopted the methodological approach of comparing actual behavior of agents (consumers; firms) to the behaviors that would be rationally computed by an agent who had perfect information and unlimited ability to process that information. But the problems faced by either households or firms are astonishingly difficult (mainly because the future is so uncertain). Of course, AIs also aim to solve extraordinarily difficult problems, and they use tools with a strong family resemblance to the tools economists have used. Our area of investigation is to compare the solutions economists have obtained using traditional methods to the solutions that would be computed by an AI trying to solve the same problem. Our expectation is that we will either find that AI solution methods are much more powerful, or that they go off the rails and fail to find the correct truly rational solution. Either finding would be interesting in the rapidly developing how-to-use-AI-in-economics literature.

      \textbf{Project Goals/Objectives and Methods}.  Our goal is to understand the differences between solutions to a canonical model that can be obtained (painfully) using tools that are known to provide numerically optimal solutions (the economists' traditional approach) to the solutions produced by AI tools.  

      As for method, we already have a case of a model -- the \cite{KrusellSmith} model -- that has been solved using both technologies,\footnote{A special issue of the \textit{Journal of Economic Dynamics and Control}, \cite{JEDCspecial}, was devoted some years ago to comparing alternative solution methods, which gives us a rich basis for comparison.} but little is known about the circumstances in which the two solutions will be similar and those in which the AI solution differs markedly from the traditional ``optimal'' solution.

     While the authors of the AI solution (\cite{MALIAR202176}) find that the results of their approach are similar to those obtained by economists' standard methods, they do not provide much insight into the reasons for, or the robustness of, this result. Until the answers to those questions are better understood, economists will be reluctant to treat AI technologies as a respectable alternative to solving models in the traditional way.

The field would greatly benefit from a proper \textit{interpretation} of the neural network's solution. That is, if the second moment of the wealth distribution \textit{isn't} relevant, but the neural network found information of the same dimensionality that \textit{is} relevant, then \textit{what is that information}? \cite{MALIAR202176} are silent on this critical question, the answer to which would inform further development of both economic theory and computational methods.  (See \cite{Reiter2010} for insightful work on this point).  That is, if we have a better idea of \textit{what to look for} based on the neural network's solution to one model, these insights can be applied to other models \textit{whether or not} a neural network is used to solve them.

      \textbf{Attractiveness for Future Funding}. Funding will flow to this area only when there is some practical wisdom about what works and what doesn't. Our aim is to build that practical wisdom, in the context of a familiar model. Once the differences between AI-based solutions and traditional solutions is understood, a wide path for future funding becomes available because a little knowledge of which approaches work better in which contexts should give funders more certainty about which projects are promising and which are not.
      
      % \noindent \textbf{Background:} A central assumption in economic modeling is that agents (decision-makers) have \textit{rational expectations} about future events. That is, while they do not know future states with certainty, they have accurate \textit{beliefs} about the \textit{distribution of events} that could occur, conditional on all information available to them in the present. Combined with the equally fundamental assumption of \textit{rational behavior}-- that each agent chooses their action (e.g.\ how much to spend on consumption vs save for the future) to optimize their preferences in some way-- the rational expectations assumption provides discipline to economic models. These assumptions ``pin down'' model predictions by restricting behavior and beliefs to be ``correct'' within the context of the model: there are infinite ways to be wrong and only one way to be right. While there is plenty of research on models of non-rational expectations (including by me in \cite{cAndCwithStickyE}) and evidence showing that human behavior \textit{cannot} be rationalized in some experimental contexts, these principles remain the standard core of modern economic theory.

%For decades, macroeconomists typically worked with \textit{representative agent} models, in which all households are agglomerated into a single \textit{representative consumer}, who is employed by the unitary \textit{representative firm}, etc. Such a reductive approach was justified by a widely held belief that heterogeneity among households or firms was of second or third order importance for aggregate, macroeconomic outcomes, as well as the practical consideration that modeling rational expectations about future outcomes is only feasible if present and future states can be represented by a reasonable number of variables. Over the past 25 years or so, the importance of the heterogeneity of household states (e.g.\ the distribution of wealth) has been taken more seriously by macroeconomists, nicely summarized in \cite{FiveGuys}.

%Properly solving a heterogeneous agents macroeconomic model is a daunting challenge, conceptually and computationally. Future prices depend on the future behavior of many agents (in the limit, a continuum), each of whose action depends on both their particular idiosyncratic circumstances and the entire distribution of states of their peers-- everyone has to know where everyone else is and predict what others are going to do. The canonical approach to this problem was presented in \cite{KrusellSmith}, who reduced households' information set to a small set of statistical measurements about the full distribution (ultimately, just the mean) and showed that the value of further information was of little consequence to individual welfare nor aggregate dynamics. Subsequent work demonstrated that this result was not universal and critically depended on the simplifying assumptions, and explored other methods for reducing the complexity of the state space while preserving solution accuracy (see e.g. \cite{Reiter2010}). Variations on the original model used by Krusell and Smith have remained the benchmark for exploring and discussing these topics, including in a special issue of the Journal of Economic Dynamics and Control (\cite{JEDCspecial}).

%More recently, researchers have begun to explore how deep-learning and neural-network approaches can be used to solve heterogeneous agents macroeconomic models. In particular, \cite{MALIAR202176} present a unified framework for solving such models using deep learning methods. Prior work usually divides the Krusell-Smith model into two complementary parts: 1) how agents should \textit{behave} conditional on the current state and their understanding of aggregate dynamics; and 2) how the macroeconomic state \textit{dynamically evolves} given how individual agents behave. The MMW paper instead solves both parts together, casting the \textit{entire model} into a form usable by a neural network (or other functional approximator), and showing that three related approaches to representing the model all yield comparable results. As usual, the benchmark model is based on \cite{KrusellSmith}, but the authors only briefly explore how their deep-learning solution systematically deviates from that produced by the traditional moment-based method. The authors conclude that incorporating more neurons in the second hidden layer (representing aggregate dynamics) is effective in improving solution accuracy, but including more moments of the state distribution is not. They explain that this is because the moments are \textit{exogenously chosen} pieces of information that \textit{aren't valuable} to agents when choosing their action, whereas the neural-network approach \textit{endogenously} finds the aggregate information that is relevant to individuals.

%\vspace{0.25cm}

%\noindent \textbf{Domain of PI}. I am well equipped to conduct this work with an AI-Informed Seed Grant.  


%\cite{MALIAR202176} report model solution times ranging from 9 minutes to 12 hours (on an ordinary laptop) depending on how many agents are tracked in the population for their method (1 to 1000). From personal experience with their variation of the Krusell-Smith model, I know that the traditional (moment-based) method takes only a few minutes to solve on ordinary hardware. The (approximately one page) discussion of this topic in the published paper does not provide any insight into the \textit{economic magnitude} of the improvement in solution accuracy that is attained by spending orders of magnitude more computational time. As a basic threshold matter, I would conduct a more thorough exploration and report of the trade-off between computation time and solution accuracy, using standard measures for the field. As you might guess, economists are interested in \textit{what's gained} when additional resources are spent.

%Beyond that straightforward analysis, the AI-Informed Seed Grant would be used to investigate a hybrid approach that lies between the traditional moment-based method from \cite{KrusellSmith} and the unified approach in \cite{MALIAR202176} that passes the entire model as a problem for the neural network to solve. Returning to the framing in the Background portion of this narrative, solving for \textit{rational behavior} conditional on beliefs is a well understood problem in computational economics; there are many ``secrets'' and ``tricks'' that have developed over time for efficiently solving such problems. Under the ``unified'' approach in \cite{MALIAR202176}, decades of specialized know-how about solving consumption-saving problems is thrown out and the neural network is left to its own devices, unguided by economic theory about its task.  In contrast, representing \textit{rational expectations} when the state space is high dimensional is \textit{not nearly} as well understood or developed. Given a long history of the states and actions of millions of agents, an economist would have some \textit{informed guesses} about how best to summarize the dynamic properties of this information, but they wouldn't necessarily be very good. To wit, Per Krusell and Tony Smith thought that the second and third moments of the wealth distribution would be relevant, but they were wrong. Parsing large datasets for the important ``hidden'' features is \textit{exactly} what deep learning methods excel at.

%\vspace{0.25cm}

%\noindent \textbf{Project Goals, Methods, and Impact:} The main task for the AI-Informed Seed Grant would be to investigate and report on a hybrid method in which the \textit{microeconomic} problem (rational behavior conditional on beliefs about aggregate dynamics) is solved by human-coded methods, drawing on decades of specialized knowledge, while the \textit{macroeconomic} component (characterizing aggregate dynamics from individual behavior) would be conducted by neural network. I believe this approach would best apply machine learning to the aspect of heterogeneous agents macroeconomics for which it is best suited, maximizing its efficiency in the solution method.

      I am particularly well positioned to accomplish the goals outlined in this proposal, for several reasons.  I am one of the leading theorists on consumption-saving models (\cite{CarrollBuffer}), as well as the developer of foundational methods for their efficient solution (\cite{CarrollEGM}). Second, I am the PI for the \href{https://econ-ark.org}{Econ-ARK} project that produces open source software for solving heterogeneous agents models (the \href{https://github.com/econ-ark/HARK}{HARK} Python package).  I already have an implementation of the \cite{MALIAR202176} paper and several implementations of the Krusell-Smith model readily at hand, and am familiar with things like the solution time each takes. Furthermore, I am well connected to the authors of \cite{MALIAR202176}, as well as to experts who are very knowledgeable in all the domains necessary to complete the project (see the budget justification for details of the help I anticipate obtaining for completing the project).

Finally, I have recently been installed as the president of the Society for Computational Economics (SCE), taking over from Lilia Maliar (of \cite{MALIAR202176}), in which role I plan to use the tools in my power to foster more research in this area. 

      
\textbf{Evidence of plan to secure follow-on funding}. 
The Econ-ARK project is currently funded by a generous corporate sponsorship from T.\ Rowe Price, and has previously received a large grant from the Sloan Foundation, to whom we are applying for a second grant.

We have drafted a \href{https://github.com/econ-ark/OverARK/blob/master/roadmap/docs/Sloan/2024-Small-Grant/TwoPageLOIforEcon-ARK_Sloan.md}{letter of intent} to the Sloan Foundation for a grant to fund further development of our existing tools for solving models of the kind described herein.  Part of our development plan includes building a structure in which it would be easy to compare alternative solution methods to the same model. We also have good reason to hope for further funding from T.\ Rowe Price, whose corporate sponsorship of the project was motivated by the fact that the core of our toolkit is the computer code for solving an optimal saving and portfolio problem for households -- something TRP (and other financial firms) are beginning to try to do themselves.

\textbf{Post-Award Requirements}.  The Econ-ARK project has recently produced a draft report for a grant previously received from the JHU Open Source Programs Office.  A review of \href{https://econ-ark.github.io/FOSSProF/}{that report} should inspire confidence in our ability to produce a high quality report when this project concludes. 

%Econ-ARK's developers are currently working on a modeling language that will allow for much greater cross-compatibility of solution methods, including machine-learning toolkits. Being able to demonstrate a proof-of-concept by connecting \texttt{HARK}'s hand-coded microeconomic solvers to a neural network that can efficiently characterize the relevant macroeconomic dynamics would make Econ-ARK \textit{significantly} more attractive to other funders.


\nocite{HARK}
\pagebreak\newpage

\begin{center}
	\textbf{Budget and Budget Justification}
\end{center}

\noindent I am requesting \$25,000 in funding from the AI-Informed Seed Grant program. These funds will primarily be spent on labor costs for two collaborators, whom I will direct and supervise:
\begin{itemize}
	\item Matthew N.\ White (JHU PhD 2014), a full time employee of Econ-ARK who was the primary developer for the \texttt{HARK} software package and has collaborated with me on heterogeneous agents macroeconomic research projects. 
	
	\item Marc Maliar (UChicago BA 2022), an aspiring economist who served as a programmer for the TensorFlow code used for \cite{MALIAR202176} and has intimate knowledge of its development. 
\end{itemize}

\noindent Marc's primary task is simply to ``onboard'' White and me, so that we are familiar with the MMW TensorFlow code structure and can reproduce its results locally. We already have our own code for a benchmark version of the Krusell-Smith model, providing an additional source of verification. I will then work with Dr.\ White to more fully characterize the nature of the differences between solutions produced by the methods.

\vspace{0.2cm}

\noindent Conservatively budgeting \$3000 of Marc Maliar's time to bring us up to speed (and for additional consulting later), and providing my own research time gratis, this leaves over six weeks of Dr.\ White's full time. Given the well bounded nature of this seed project, our ``hot start'' on the work, and the importance of this project in attracting further funding from (e.g.) the Sloan Foundation and T.\ Rowe Price, I am quite confident that we can produce a thorough report in the short time frame.


%Tying these last two together into the fourth and final reason: this summer Econ-ARK employed the Maliars' son Marc, who actually helped them write the TensorFlow code for that paper. Developing the hybrid method that I have outlined here is eminently feasible for me because I already am the PI for an expertly developed microeconomic solution codebase \textit{and} direct access to the authors and programmers for the seminal paper that I seek to improve upon. The tools are there, and the AI-Informed Seed Grant would provide the means for me to put them together.

\newpage

\begin{singlespace}
    \bibliographystyle{mnwteststyle}
	\bibliography{AIseed}
\end{singlespace}

\end{document}
