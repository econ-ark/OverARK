
       \documentclass[12pt,pdftex,letterpaper]{article}
            \usepackage{setspace}
            \usepackage[dvips,]{graphicx} %draft option suppresses graphics dvi display
%            \usepackage{lscape}
%            \usepackage{latexsym}
%            \usepackage{endnotes}
%            \usepackage{epsfig}
            \usepackage{amsmath}
%           \singlespace
            \setlength{\textwidth}{6.5in}
            \setlength{\textheight}{9in}
            \addtolength{\topmargin}{-\topmargin} 
            \setlength{\oddsidemargin}{0in}
            \setlength{\evensidemargin}{0in}
            \addtolength{\headsep}{-\headsep}
            \addtolength{\topskip}{-\topskip}
            \addtolength{\headheight}{-\headheight}
            \setcounter{secnumdepth}{2}
%            \renewcommand{\thesection}{\arabic{section}}
            % \renewcommand{\footnote}{\endnote}
            \newtheorem{proposition}{Proposition}
            \newtheorem{definition}{Definition}
            \newtheorem{lemma}{lemma}
            \newtheorem{corollary}{Corollary}
            \newtheorem{assumption}{Assumption}
            \newcommand{\Prob}{\operatorname{Prob}}
            \clubpenalty 5000
            \widowpenalty 5000
            \renewcommand{\baselinestretch}{1.1}
            \usepackage{amsmath}
            \usepackage{amsthm}
            \usepackage{amsfonts}
            \usepackage{amssymb}
            \usepackage{bbm}
            \newcommand{\N}{\mathbb{N}}
			\newcommand{\R}{\mathbb{R}}
			\newcommand{\E}{\mathbb{E}}
			\newcommand{\der}[2]{\frac{\text{d}#1}{\text{d}#2}}
			\newcommand{\pd}[2]{\frac{\partial#1}{\partial#2}}
			
			\setlength{\parindent}{0pt}
			\setlength{\parskip}{1em}
						
\begin{document}
	\pagestyle{empty}
	
James:

If you recall, we met briefly in [LOCATION] in [MONTH YEAR]. At that time, the newly founded Econ-ARK project had recently received a generous grant from the Alfred P.\ Sloan Foundation to develop the \texttt{HARK} toolkit, a software package for solving, simulating, and estimating heterogeneous agents macroeconomics models. Using NumFocus as our fiscal sponsor, we have since received additional funding from other sources (including the Think Forward Institute and T.\ Rowe Price as a ``no strings attached'' corporate sponsor) to continue the work and expand the range of models offered in the \texttt{HARK} package, allowing us to achieve our original set of goals.

Over the past seven years, the combination of our experience producing \texttt{HARK}, feedback we have received since its inception, and relatively recent developments in other software packages have led the Econ-ARK team to conclude that a new software package and modeling schema is needed to further advance the field and fully realize our vision. I am writing to you to inquire about seeking potential funding from Schmidt Futures for this endeavor.

By way of background, economists have increasingly developed dynamic models of the decisions of individual agents (i.e.\ households or firms) that are intended to be interpreted \textit{quantitatively} rather than only \textit{qualitatively}. Instead of merely characterizing the general properties of the outcomes predicted by a particular theoretical model, researchers instead seek to quantitatively evaluate its performance in matching past observed outcomes, and to generate quantitatively plausible predictions of the effects of hypothetical future events and policies. This approach to economics is often referred to as structural modeling. When applied to macroeconomic topics, structural modeling involves generating aggregate outcomes as the accumulation of the individual decisions of many agents, each of whom may face different circumstances idiosyncratic to themselves. Hence this subfield is known as heterogeneous agents macroeconomics (``HA macro''), in contrast to the more traditional representative agent (``RA'') approach in which idiosyncratic heterogeneity among households and firms is assumed away.

Modern models in both HA macroeconomics and structural microeconomics include stochastic processes that realistically reflect the large uncertainties inherent to the agents' problem-- potentially large shocks to their productivity, demand for their labor (or output), health, etc. The methods for solving models with such features are computationally complex, and are not usually taught as part of graduate economics training-- it takes years of experience to understand how they should be applied, and how to interpret (and correct) their apparent failures and shortcomings.

Attached to this email, please find a brief letter of interest describing our proposed work, as well as background information on why the project's output is sorely needed and would be a great boon to several constituencies. Thank you very much, and I hope you are well.


\end{document}