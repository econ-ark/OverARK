
       \documentclass[11pt,pdftex,letterpaper]{article}
            \usepackage{setspace}
            \usepackage[dvips,]{graphicx} %draft option suppresses graphics dvi display
%            \usepackage{lscape}
%            \usepackage{latexsym}
%            \usepackage{endnotes}
%            \usepackage{epsfig}
%           \singlespace
            \setlength{\textwidth}{6.5in}
            \setlength{\textheight}{9in}
            \addtolength{\topmargin}{-\topmargin} 
            \setlength{\oddsidemargin}{0in}
            \setlength{\evensidemargin}{0in}
            \addtolength{\headsep}{-\headsep}
            \addtolength{\topskip}{-\topskip}
            \addtolength{\headheight}{-\headheight}
            \setcounter{secnumdepth}{2}
%            \renewcommand{\thesection}{\arabic{section}}
            % \renewcommand{\footnote}{\endnote}
            \newtheorem{proposition}{Proposition}
            \newtheorem{definition}{Definition}
            \newtheorem{lemma}{lemma}
            \newtheorem{corollary}{Corollary}
            \newtheorem{assumption}{Assumption}
            \newcommand{\Prob}{\operatorname{Prob}}
            \clubpenalty 5000
            \widowpenalty 5000
            \renewcommand{\baselinestretch}{1.1}
            \usepackage{amsmath}
            \usepackage{amsthm}
            \usepackage{amsfonts}
            \usepackage{amssymb}
            \usepackage{bbm}
            \usepackage{hyperref}
            \newcommand{\N}{\mathbb{N}}
			\newcommand{\R}{\mathbb{R}}
			\newcommand{\E}{\mathbb{E}}
			\newcommand{\der}[2]{\frac{\text{d}#1}{\text{d}#2}}
			\newcommand{\pd}[2]{\frac{\partial#1}{\partial#2}}
			\setlength{\parindent}{0pt}
			\setlength{\parskip}{0.9em}
						
\begin{document}
	
\begin{singlespace}
	Prof Christopher D.\ Carroll \hfill James Savage\\
	Wyman Park Building 590 \hfill Schmidt Futures\\
	Johns Hopkins University \hfill 155 W 23rd St\\
	Baltimore, MD 21211 \hfill New York, NY 10011
	
	\vspace{0.2cm}
	
	January 24, 2024
	
\end{singlespace}

\vspace{0.3cm}

James:

When we last spoke in 2017, the newly founded \href{http://www.econ-ark.org}{Econ-ARK} project had recently received a generous grant from the \href{https://sloan.org}{Alfred P.\ Sloan Foundation} to develop the \href{https://github.com/econ-ark/HARK}{HARK toolkit}, a software package for solving, simulating, and estimating heterogeneous agents macroeconomics models. Using \href{https://numfocus.org/}{NumFocus} as our fiscal sponsor, we have since received additional funding from other sources (including the \href{https://inomics.com/institution/think-forward-initiative-1258337}{Think Forward Initiative} and \href{https://www.troweprice.com/en}{T.\ Rowe Price} as a ``no strings attached'' corporate sponsor) to continue the work and expand the range of models offered in the HARK package, allowing us to achieve our original set of goals. With seven years of experience underneath our belt, we now see that our next steps involve developing a \textit{language} for expressing dynamic structural models that can specify numeric methods, describe simulation procedures, and generate model output. I am writing to you to inquire about seeking funding from Schmidt Futures for this endeavor.

To fix ideas about the language, a structural model would be formalized as a model file: a set of statements or directives characterizing its components, similar to how a \href{https://www.dynare.org/}{Dynare} user declares the aggregate budget constraint as \texttt{c+k=k\^{}theta*A}, or a \href{https://www.econforge.org/dolo}{dolo} user expresses an intertemporal arbitrage condition as \texttt{1 = beta*(c/c(1))\^{}(sigma)*(1-delta+rk(1))}. The language is intended to provide a common format for describing dynamic structural models (with a broad variety of features) that will be widely adopted for conveying model content in a human- and machine-readable way, but is independent of the code to actually solve and implement the model. To that end, we are also developing a software platform that interprets model statements in the new language, generates code for solving and simulating the model (with the chosen numeric methods), and allows the user to interactively build a structural model and examine its solution and output.

A tool of this kind already exists for classical representative agent macroeconomic models-- the widely adopted \href{https://www.dynare.org/}{Dynare} package-- but it cannot be used or extended for more modern models with non-trivial heterogeneity and idiosyncratic risk. When discussing or presenting our work on the HARK package, other economists often interpret it through the lens of Dynare, eagerly hopeful that we have built ``Dynare for heterogeneous agents''. This has left us no doubt that there is a large demand for exactly this sort of tool. With our prior experience implementing \texttt{HARK}, and given recent advances in other software tools, we are now prepared to design and create it.

Prior to beginning work on the new platform, we conducted a thorough search of \textit{other} academic fields, investigating whether a general dynamic modeling schema has already been developed. Despite considerable effort, we found that there is no comparable or related project that could be adapted or expanded for our purposes. Rather, we found that the universe of modeling- and optimization-adjacent software is both diverse and diffuse: a collection of useful but \textit{unconnected} software tools. The lack of a common platform for representing dynamic models and calling tools for handling subsets of them is akin to the lack of cohesion among the various artificial intelligence (AI) and deep learning toolkits that have recently been developed. That is, researchers who want to use multiple AI tools must write their own code to link to each one individually, rather than there being any kind of common interface. Our project thus has potentially large spillovers.

In addition to the technical value of this work, it will also significantly improve the transparency and replicability of structural economics research. While it is increasingly common for researchers to publicly archive their project code, a description of the numeric methods used is usually omitted from the final published paper; this information is (at best) relegated to an online appendix or (at worst) not documented anywhere outside of the code itself. Moreover, unlike the conventions of mathematically expressing a system of statements to compose a theoretical model, there is no straightforward and complete way to convey such numeric methods and details, even if an economist were so inclined. Even worse, there is no formal relationship between the model as expressed \textit{on paper} and the problem as solved \textit{in code}-- the academic refereeing system focuses deeply on the economics and relies on trust with respect to the numerics. Indeed, there are famous examples of papers that have been published based on their strong economic content, but whose quantitative (and sometimes qualitative) results were later discovered to be based on errant code.

Our proposed modeling language seeks to rectify these systemic issues with the workflow of economic research that uses dynamic structural models. If a model specification file is used to generate a numeric solution and model output, a reader or evaluator can be confident that the model presented on paper matches its execution in code. Furthermore, our language will include a format for specifying the methods used to solve the model numerically, transparently conveying this information alongside the ``pure'' mathematical content of the model. The software platform can thus act as a vehicle for evaluating the performance of a numeric solution to a theoretical model.

Our proposed platform will fill a critical gap in the toolset available to structural economists, accelerating the development of models on the frontier of economic research, improving the verifiability of numeric output from such models, and improving communication and collaboration among researchers. In addition to academic work, the platform would be of significant use both to governments (including central banks and financial regulators) in conducting prospective analyses of potential policy actions, and to private financial institutions who wish to make decisions or provide advice that is informed by a structural model (e.g.\ a model of optimal retirement savings). In developing the language and the software platform, we plan to seek out input from a variety stakeholders to ensure that their modeling needs are met.

We are very interested in beginning a dialog with you about potential funding from Schmidt Futures. Thank you very much for your time and consideration in this matter.

\vspace{0.5cm}

{\parskip=2pt Sincerely,

\includegraphics[scale=0.7]{CDCsignature.jpg}

Christopher D.\ Carroll,\\ Professor of Economics}

\end{document}


As brief background, economic models are usually expressed with a combination of mathematical and natural language: most of the content can be concisely conveyed as a series of equations, inequalities, and other mathematical statements, while a few additional details are provided in plain English outside of the formal statement. This representation of the theoretical model is intended to be readily understood by a human reader. Unlike basic economic models used for pedagogy, which can be solved algebraically and analyzed intuitively, modern dynamic structural models can be solved only \textit{approximately} using numeric methods.