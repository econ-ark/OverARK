
       \documentclass[11pt,pdftex,letterpaper]{article}
            \usepackage{setspace}
            \usepackage[dvips,]{graphicx} %draft option suppresses graphics dvi display
%            \usepackage{lscape}
%            \usepackage{latexsym}
%            \usepackage{endnotes}
%            \usepackage{epsfig}
%           \singlespace
            \setlength{\textwidth}{6.5in}
            \setlength{\textheight}{9in}
            \addtolength{\topmargin}{-\topmargin} 
            \setlength{\oddsidemargin}{0in}
            \setlength{\evensidemargin}{0in}
            \addtolength{\headsep}{-\headsep}
            \addtolength{\topskip}{-\topskip}
            \addtolength{\headheight}{-\headheight}
            \setcounter{secnumdepth}{2}
%            \renewcommand{\thesection}{\arabic{section}}
            % \renewcommand{\footnote}{\endnote}
            \newtheorem{proposition}{Proposition}
            \newtheorem{definition}{Definition}
            \newtheorem{lemma}{lemma}
            \newtheorem{corollary}{Corollary}
            \newtheorem{assumption}{Assumption}
            \newcommand{\Prob}{\operatorname{Prob}}
            \clubpenalty 5000
            \widowpenalty 5000
            \renewcommand{\baselinestretch}{1.1}
            \usepackage{amsmath}
            \usepackage{amsthm}
            \usepackage{amsfonts}
            \usepackage{amssymb}
            \usepackage{bbm}
            \usepackage{hyperref}
            \newcommand{\N}{\mathbb{N}}
			\newcommand{\R}{\mathbb{R}}
			\newcommand{\E}{\mathbb{E}}
			\newcommand{\der}[2]{\frac{\text{d}#1}{\text{d}#2}}
			\newcommand{\pd}[2]{\frac{\partial#1}{\partial#2}}
			\setlength{\parindent}{0pt}
			\setlength{\parskip}{0.9em}
						
\begin{document}
	
\begin{singlespace}
	Prof Christopher D.\ Carroll \hfill Danny Goroff\\
	Wyman Park Building 590 \hfill The Alfred P. Sloan Foundation\\
	Johns Hopkins University \hfill %155 W 23rd St
        \\
	Baltimore, MD 21211 \hfill %New York, NY 10011
	
	\vspace{0.2cm}
	
	March 9, 2024
	
\end{singlespace}

\vspace{0.3cm}

Danny (and other interested parties):

When the \href{http://www.econ-ark.org}{Econ-ARK} project's \href{generous founding grant}{https://sloan.org/grant-detail/8071} from \href{https://sloan.org}{Sloan} was nearing its end, your emphasis that Sloan's ambition was to fund the startup period of projects that would eventually be self-sustaining was something we took seriously.  So seriously, in fact, that we did not apply to Sloan for a renewal of the grant. 

That was by no means a signal that the project had been abandoned. We have since received about \$750K in additional funding from other sources (including the \href{https://inomics.com/institution/think-forward-initiative-1258337}{Think Forward Initiative} and \href{https://www.troweprice.com/en}{T.\ Rowe Price} as a ``no strings attached'' corporate sponsor through NumFocus) to continue that work and expand the range of tools and models offered in the HARK package.  We have now largely achieved our original set of goals.

With this experience beneath our belt, we now see that the next steps involve developing a \textit{language} for expressing dynamic structural models.  Such a language would specify computational  methods, describe simulation procedures, and generate model output. I am writing to you to ask whether Sloan might be interested in funding this endeavor.

To fix ideas, a structural model would be formalized as a model file: a set of statements characterizing its components, like how a budget constraint might be captured with a simple equation saying that assets are whatever is left of the consumer's money after consumption: \texttt{a=m-c} (of course much more sophisticated mathematical propositions would also be accommodated).  The language (or perhaps a better term would be schema) will provide a common format for describing dynamic structural models and will convey model content in a human- and machine-readable way that is independent of the code to actually solve and implement the model. As we develop the schema, we would simultaneously build a software platform to parse the model file into the implied code (though nothing would prevent someone else from inventing a better software platform in the future). %To that end, we are also developing a software platform that interprets model statements in the new language, generates code for solving and simulating the model (with the chosen numeric methods), and allows the user to interactively build a structural model and examine its solution and output.

%\href{https://www.dynare.org/}{Dynare} or \href{https://www.econforge.org/dolo}{dolo} user declares the budget constraint with a statement something like \texttt{c+k=k\^{}theta*A}.
%A tool of this kind already exists for classical representative agent macroeconomic models-- the widely adopted 

% , or a \href{https://www.econforge.org/dolo}{dolo} user expresses an intertemporal arbitrage condition as \texttt{1 = beta*(c/c(1))\^{}(sigma)*(1-delta+rk(1))}. 

The \href{https://www.dynare.org/}{Dynare} package provides a prototype example of the \textit{kind} of thing we have in mind, but limitations in its syntax and specification mean that it cannot be used or extended for the kinds of models that are now \textit{de rigeur} in both micro and macro modeling.  When explaining our work on HARK, we have found that other economists often interpret it through the lens of Dynare, with the hope that we have already built ``Dynare for heterogeneous agents.'' This has left us no doubt that there is a large demand for exactly such a tool. With our prior experience implementing HARK, and given recent advances in other software tools, we are now prepared to design and create it.

Prior to beginning work on the new platform, we conducted a thorough search of \textit{other} academic fields, investigating whether a general dynamic modeling schema has already been developed elswhere. Having explored all the nooks and crannies of the internet, we are confident that there is no comparable or related project that could be adapted or expanded for our purposes. We found that the universe of modeling- and optimization-adjacent software is both diverse and diffuse: We found many of the building blocks necessary to accomplish our goal, but no schemes for putting the building blocks together into anything like what we need. The lack of a common platform for representing dynamic models is akin to the lack of cohesion among the various artificial intelligence (AI) and deep learning toolkits that have recently been developed. %Translated into that context, the AI equivalent would be a language that described the AI problem to be solved in a platform-independent way, allowing a user then to solve exactly the same model with each of the competing AI tools.  (The problem is even more ambitious in that context than in ours, but what we accomplish might be a good stepping stone toward a platform-independent AI tool.)

%That is, researchers who want to use multiple AI tools must write their own code to link to each one individually, rather than there being any kind of common interface. Our project thus has potentially large spillovers.

This work will also significantly improve the transparency and replicability of structural economics research. Behind closed doors, everyone who works on these kinds of models admits (and laments) what they know to be true: Everyone's results depend on a host of ancillary assumptions -- how many gridpoints to use, how many agents to simulate, etc.  While it is now expected that researchers will publicly archive their code, for many projects the code might as well be written in Klingon (so far as accessibility and transparency and replicability are concerned).  Even worse, there is no direct relationship between the model as expressed \textit{on paper} and the problem as solved \textit{in code}-- the academic refereeing system focuses deeply on the economics of the abstract math, and relies on trust with respect to the numerics. Indeed, there are famous examples of papers that have been published based on their strong economic content, but whose quantitative (and sometimes qualitative) results were later discovered to be wrong as a result of coding errors.


%The fact that everyone writes their own code means that it usually infeasible to meaningfully test whether a published paper's results are robust to the authors' arbitrary choices.  (There are a few examples where the experiment has been tried and the results have not replicated).

%Moreover, even if an economist were so inclined, there is no standard for how the computational instantiation of the model should be described.



Our proposed modeling language aims to rectify these systemic issues with the workflow of economic research. If a model specification file is used to generate a numeric solution and model output, a reader or evaluator can be confident that the model presented on paper matches its execution in code. Furthermore, our language will include a format for specifying the methods used to solve the model numerically, transparently conveying this information alongside the ``pure'' mathematical content of the model. The software platform can thus act as a vehicle for evaluating the performance of a numeric solution to a theoretical model.

Our proposed platform will accelerate the development of models on the frontier of economic research, allow for the verifiability of numeric output from such models, and improve communication and collaboration among researchers. In addition to academic work, the platform would be of significant use both to governments (including central banks and financial regulators) in conducting prospective analyses of potential policy actions, and to private financial institutions who wish to make decisions or provide advice that is informed by a structural model (e.g.\ a model of optimal retirement savings). In developing the language and the software platform, we will seek out input from a variety stakeholders to ensure that their modeling needs are met.

Our sense is that, among potential funders, Sloan would be the best fit.

Thanks for your time and consideration in this matter.

\vspace{0.5cm}

{\parskip=2pt Sincerely,

  \includegraphics[scale=0.7]{/Volumes/Data/Personal/Identity/CDCSignatures/CDCSignature.jpg}
  
Christopher D.\ Carroll, Professor of Economics, Johns Hopkins University}

\end{document}


As brief background, economic models are usually expressed with a combination of mathematical and natural language: most of the content can be concisely conveyed as a series of equations, inequalities, and other mathematical statements, while a few additional details are provided in plain English outside of the formal statement. This representation of the theoretical model is intended to be readily understood by a human reader. Unlike basic economic models used for pedagogy, which can be solved algebraically and analyzed intuitively, modern dynamic structural models can be solved only \textit{approximately} using numeric methods.
